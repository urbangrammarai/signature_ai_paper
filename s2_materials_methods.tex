\section{Materials and Methods}
\label{sec:matmet}

\subsection{Materials}

\subsubsection{British Spatial Signatures}

% 500 words

\subsubsection{Sentinel 2 imagery}

% 250 words

\subsection{Methods}

% 250 to explain the overarching experiments

\subsubsection{Chip size}

% 500 words

\subsubsection{Data (spatial) augmentation}

% Sliding

% 250 words

\subsubsection{Model architecture}

% 500 words

% Overall content of the section

% NN architecture: not of particular interest, hence we pick EfficientNet
% (but we show in appendix a brief comparison why)
Appendix \ref*{sec:appendixA} shows a brief comparison of several standard neural network architectures.

% We define our challenge as an image classification task and use competing
% alternatives to explore which one performs best. Each of them imply
% geographically interesting trade-off's

% Standard image classification

% Multi-output regression

% Spatial modelling of probabilities

\subsubsection{Performance metrics}

% 500 words

% Traditional non-spatial

% Explicitly spatial metrics
%% Why
%% Which ones
%% Why those? (ideally link to Miguel's paper suggestions)

\subsubsection{Summarizing experiments}

% 250 words

% explanation of regression approach