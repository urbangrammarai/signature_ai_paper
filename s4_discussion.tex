% Summarise results
The results can be summarised in four dimensions.
%% Architecture dimension (BIC, SIC, MOR)
The first dimension tested is the way chip sampling and a related architecture of used
CNN. It seems clear that the baseline image classification is limited, and either the
sliding approach to minimise the disbalance of sample size per class or multi-output
regression shall be preferred in a use case like signature detection. Of the two,
multi-output regression even seems to be better, and one of the reasons could be its
ability to capture co-location. While BIC and SIC-based models have no information on
the geographical relationship between neighbouring signature types, MOR directly
captures these as chips often cross multiple signature types. This behaviour is unique
to geographical problems. Traditional image classification tasks are not able to encode
\textit{distance} between two types.
%% Chip size dimension
The second dimension is chip size. Except for Join Counts statistics, we see a positive
relationship between model performance and the extent our chips cover. It is an expected
outcome as the larger the chip is, the more information it contains. However, we cannot
blindly follow \textit{larger is better} logic as signature types are composed of
granular geometries, and we see a sampling issue when the chip size grows. While that
can be partially mitigated by using MOR, it needs to be considered in model
architecture.
%% Modelling dimension
Another dimension looks at the value of modelling on top of probabilities coming from
neural networks. The results indicate that there is a value of the modelling step as the
maximum probability option tends to underperform both logit models and histogram-based
boosted classifiers. While the difference between logit and HGBC is not always
significant, some results suggest that the non-linear nature of HGBC provides a better
outcome than linear logit models.
%% W dimension
The last dimension focuses on the inclusion of spatial lag in the modelling step as a
geographically-explicit method of capturing the context of each chip. This has one of
the most consistent results indicating the models that exclude spatial lag have worse
performance than those that include it. Yet again, this step would not be possible in a
traditional image classification context where two samples have no distance from each
other.
% What is the best
Combining all dimensions, we can assume that the optimal model for the detection of spatial
signatures from Sentinel 2 satellite imagery should define CNN for the multi-output
regression problem based on larger chip size and passing the output to non-linear
probability modelling with a spatial lag component.

% Ability to capture signatures % We can do some nicely, some worse. What is next?
That said, we cannot assume that even the best model will perform evenly across all 12
signature types. The within-class performance metrics indicate that some classes on the
extreme sides of the urban-rural dimension are easier to detect. That is not surprising
as both \textit{Urbanity} and \textit{Wild countryside} signature types are unique,
while a difference between \textit{Dense residential neighbourhoods} and
\textit{Connected residential neighbourhoods} that are visible on the satellite imagery
is much more subtle. It is also common that some of the classes are easier to
distinguish than others \cite{zanaga_daniele_2021_5571936, karra2021global}. However,
any model deployed for periodical updates of signature classification will have to deal
with this limitation.

% limits
%% Sentinel 2 resolution
%% Chip sample imbalance
%% explain why segmentation is not used and tested
The experiments presented in this article focus on specific target data represented by
spatial signatures. Because the signatures are designed to capture the structure of urban
environments, the behaviour of spatial components in the modelling pipeline may differ
when target data of a different nature are used. However, we would argue that the
principle still holds in most cases as the uniqueness of satellite data in the image
classification is undeniable and will always offer specific solutions not generally
available when the task is to distinguish a cat from a dog.

Since the article was restricted to the use of open data at every step, the best
resolution of satellite imagery was 10 meters per pixel offered by the Sentinel 2
mission. That poses some challenges because such a resolution limits the amount of
information we can capture on a small area and may oversimplify urban environments that
are naturally more granular in their patterns than what 10mpp can capture. Further
research should explore the performance differences when commercial very-high-resolution
imagery is used instead.

The combination of signatures reflecting small-scale urban types and a relatively coarse
resolution leads to another limitation this work faces - the struggle to sample chips
in a balanced manner. This is most prominent in the baseline image classification
problem, where no pixels are shared among chips and all chips need to be exclusive to a
single signature type. The issue is alleviated by class weights in the neural network
architecture, but such a solution is not optimal.

When selecting the architecture of neural networks, we have intentionally excluded image
segmentation. While it seems like an ideal candidate for the task at hand, there are
several reasons it was excluded. The first has to do with the spatial signatures and the
nature of the boundaries between individual types. While the dataset from
\cite{fleischmann2022geographical} delineates them with hard boundaries when one cell
is a type A and the neighbouring one a type B, the reality is not that simple, and these
boundaries should be treated more as a fuzzy boundary. There is very rarely a hard
switch between one type of urban environment and the other one. In many cities, two
types tend to form a transition on the edges where neither is dominant. A situation like
this is very challenging for image segmentation. The second reason is that image
segmentation, having a prediction for individual pixels, would not allow us to use the
second part of the method and test the effect of spatial lag in modelling. The only
way of doing that would be to run the experiment on a pixel level which would be
extremely computationally expensive, hence challenging to reproduce. We believe that the
method that can be run on a local machine is in the end, more valuable than the one
requiring a high-performance cluster.


%% What do we make of it in terms of geography
%%% Reiterate the point on the relevance of geography and how our results
%%% support that view and introduce explicitly-spatial/geographical ways to
%%% improve CV models for spatial imagery
Is geography relevant in image classification problems, then? The results presented
above suggest so. An introduction of geographical methods to improve image
classification models based on spatial imagery proves to be beneficial and makes use of
unique - spatial - dimension offers. It requires moving beyond traditionally used
pre-trained models than have no sense of adjacency of individual chips/samples. We need
to make a step towards merging our GIS knowledge with the one that lies in the field of
AI, often based in departments of computer science rather than geography.

% finish with a very general note on the amount of sat data coming and the need to make
% sense of it
While satellite imagery and neural networks have been around for some time already, we
are just entering the era of an increasing abundance of satellite-based data. What used
to be reserved for national agencies and international consortia is becoming a domain of
commercial subjects. Research in the remote sensing area will face not a lack of
available data but the opposite. We may find ourselves in a situation where a vast
amount of data streams will come our way, but we will struggle to make sense of it. We
believe that the research presented in this article helps in finding our way through.