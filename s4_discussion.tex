\section{Discussion}
\label{sec:discussion}

% 1 000 words

% Summarize results
The results can be summarised in four dimensions.
%% Architecture dimension (BIC, SIC, MOR)
The first dimension tested is the way chip sampling and a related architecture of used
CNN. It seems clear that the baseline image classification is limited, and either the
sliding approach to minimise the disbalance of sample size per class or multi-output
regression shall be preferred in a use case like signature detection. Of the two,
multi-output regression even seems to be better, and one of the reasons could be its
ability to capture co-location. While BIC and SIC-based models have no information on
the geographical relationship between neighbouring signature types, MOR directly
captures these as chips often cross multiple signature types. This behaviour is unique
to geographical problems. Traditional image classification tasks are not able to encode
\textit{distance} between two types.
%% Chip size dimension
The second dimension is chip size. Except for Join Counts statistics, we see a positive
relationship between model performance and the extent our chips cover. It is an expected
outcome as the larger the chip is, the more information it contains. However, we cannot
blindly follow \textit{larger is better} logic as signature types are composed of
granular geometries, and we see a sampling issue when the chip size grows. While that
can be partially mitigated by using MOR, it needs to be considered in model
architecture.
%% Modelling dimension
Another dimension looks at the value of modelling on top of probabilities coming from
neural networks. The results indicate that there is a value of the modelling step as the
maximum probability option tends to underperform both logit models and histogram-based
boosted classifiers. While the difference between logit and HGBC is not always
significant, some results suggest that the non-linear nature of HGBC provides a better
outcome than linear logit models.
%% W dimension
The last dimension focuses on the inclusion of spatial lag in the modelling step as a
geographically-explicit method of capturing the context of each chip. This has one of
the most consistent results indicating the models that exclude spatial lag have worse
performance than those that include it. Yet again, this step would not be possible in a
traditional image classification context where two samples have no distance from each
other.
% What is the best
Combining all dimensions, we can assume that the optimal model for the detection spatial
signatures from Sentinel 2 satellite imagery should define CNN for the multi-output
regression problem based on larger chip size and passing the output to non-linear
probability modelling with a spatial lag component.

% Ability to capture signatures % We can do some nicely, some worse. What is next?
That said, we cannot assume that even the best model will perform evenly across all 12
signature types. The within-class performance metrics indicate that some classes on the
extreme sides of the urban-rural dimension are easier to detect. That is not surprising
as both \textit{Urbanity} and \textit{Wild countryside} signature types are unique,
while a difference between \textit{Dense residential neighbourhoods} and
\textit{Connected residential neighbourhoods} that are visible on the satellite imagery
are much more subtle. It is also common that some of the classes are easier to
distinguish than others \cite{zanaga_daniele_2021_5571936, karra2021global}. However,
any model deployed for periodical updates of signature classification will have to deal
with this limitation.

%% What do we take of it in terms of geography
%%% Reiterate the point on the relevance of geography and how our results
%%% support that view and introduce explicitly-spatial/geographical ways to
%%% improve CV models for spatial imagery


% limits
%% Sentinel 2 resolution
%% Chip sample imbalance
%% explain why segmentation is not used and tested

% finish with a very general note on the amount of sat data coming and a need to make sense of it