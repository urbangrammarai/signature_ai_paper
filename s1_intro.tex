\section{Introduction}
\label{sec:intro} % ca. 1500 words

% Urban form and function
%% Important to understand
The way in which different urban functions are arranged within space, and the forms these
give rise to, are important to understand how cities work, how they
interact with the human and environmental systems that create them, and how
policy can effectively intervene.
%% Why? Encodes history, conditions the future
Urban form and function matter, at least, for two reasons \citep{dab_mf_2021a}:
first because cities use both to encode their history; and second because,
once in place, the physical layout of functions within a city condition how it
can and will develop in the future.
%% Key to understanding is measurement
A key requirement to understand form and function in cities is adequate
measurement, which implies detailed, consistent, and scalable
characterisations that can be updated frequently over time. These
characteristics then allow not only to observe detail, but to see it unfold
both over space and time.
%% Rare and incomplete currently, of detailed, scalable and consistent, pick any two
There is a large literature measuring these phenomena, and it is relatively
common to find any two of those characteristics (i.e., detailed and consistent,
consistent and scalable, and detailed and scalable) present in a given piece
of work.
%% Recent developments are changing this --> spatial signatures
Research bringing the three together is still rare, although some is
emerging (e.g., \citealp{fleischmann2022geographical}) thanks to the confluence of better data, open
source software, and cheap computing power.
%% Detailed measurement is expensive in terms of time, effort, and data requirements
Still, generating detailed, consistent, and scalable classifications of urban
form and function is an expensive process that is difficult to refresh
regularly because most of the underlying data sources only see updates
infrequently.

% The promise of satellites
%% A promising solution is to _supplement_ detailed measurements with satellite imagery
A promising option to improve the frequency of these classifications is
satellite imagery.
%% Satellite has radically improved in the last years, and it is set to continue on that technological path
Satellite technology has radically increased and improved the amount of
imagery available on the Earth, and shows no signs of slowing down.
%% At the same time, the algorithms to process satellite have also seen a revolution in the last ten years (deep learning)
More and better imagery has been complemented with the rise of new computer
vision algorithms, such as deep learning \citep{lecun2015},
that allow to extract more value from the same amount of data; and the
availability of computing power that makes it possible to deploy them cheaply
without the steep learning curve required only a few years ago.
%% These two trends converge in making possible things with satellite that was unthinkable a few years ago
The convergence of these two trends in remote sensing is unlocking
achievements that even very recently seemed beyond the realm of possibility.
%% Such as measuring UFF using and end-to-end open pipeline (data and code)
One such area is the use of remote sensing and satellite technology to decode
complex patterns in urban landscapes, such as the spatial signature of
different types of form and function.
%% Open is important because it multiplies the options of what is possible to do with outputs
Just as importantly, many of these advances are being built atop
technology developed under open licenses that allow to further build on them, freely
redistributing downstream outputs.

            % Lit. review %
% Satellites for cities
The use of satellite technology for measuring different aspects of urban
environments is by no means new.
%% Mostly through Remote Urban Sensing
Much of the present work falls within the
broad category of urban remote sensing \citep{rashed2010remote, weng2018urban,
yang2021urban}. In fact, the promise of using remote sensing data to decode
the complexity of urban structure has long been recognised (e.g.,
\citealp{longley2002geographical}).
%% The vast majority is as supervised object detection --> include building footprints
Much of the work in this area has traditionally focused on identification of
individual geographic features, such as building footprints (e.g.,
\citealp{microsoft2019}) or trees (e.g., \citealp{ke2011review}). More
recently, the field has started to pay increasing attention to the use of
modern algorithms such as deep learning \citep{lai2021deep}, and attempting to
map more complex patterns that involve bundles of features rather than a single
one (e.g., \citealp{kuffer2021mapping}).
%% And as Land Use and Land Cover --> Include recent examples (ESRI land cover, Google Dynamic World, etc.)
On the adjacent domain of land use and land cover mapping, recent advances
have shown the potential of using frequently updated, open satellite data in
combination with modern computer vision to effectively map land cover globally
in quasi continuous ways
(e.g., \citealp{karra2021global, brown2022dynamic}; see \citealp{venter2022global} for a
detailed comparison of some of the most novel data products in this realm).

% Satellites for urban form and function
%% Much less on recognising composite patterns (e.g., UFF) instead of single objects or uses
While most of the efforts in urban remote sensing have focused on the
identification of individual features or single uses, much less work has been
directed at decoding patterns that involve several features and/or uses to be
identified.
%% In some ways, this is more complicated but, in others, maybe not (plus we have much better tools now!)
In some ways, the jump from the simpler goal of identifying one object or a
single use to detecting a pattern that involves a particular bundle of them is
not without its challenges and shortcomings \citep{wang2022knowledge}.
But, given the performance of modern algorithms, and the increase in
resolution and quality of even openly available imagery, realising this goal
is starting to become possible.
%% Much focused on Local Climate Zones and slums
There are two areas that have received most of the attention in this context.
One revolves around the prediction of Local Climate Zones (LCZs,
\citealp{stewart2012}). LCZs are a set of pre-defined classes of urban
fabric originally developed for the study of the urban heat island effect. A
growing body of literature has focused on developing more exhaustive and
sophisticated models to extract these classes from satellite imagery (e.g.,
\citealp{koc2017mapping, wang2018mapping, liu2020local, taubenbock2020, zhou2021parcel, zhou2022deep}).
% Slums
The second one is focused on one particular type of urban form and function
that is mostly found in regions which are typically data scarce: informal
settlements, or urban slums. For the interested reader, \cite{slums2016}
provides an excellent starting point.
% A bit of morphometrics
Although much more in its infancy, a nascent area of interest is growing
around using imagery to decode urban form (e.g., \citealp{04f9ab8f6c714010ac39b58230f59d85}).

% Deep Learning Architectures
A common element of the recent advances reviewed above is the use of deep
convolutional neural networks to perform the task of interest (i.e.,
classification/segmentation/recognition) from satellite imagery.
% From scratch
Some studies, particularly those with sufficient data and computation available, train
networks from scratch. These involves sometimes building a bespoke
architecture (e.g., \citealp{othman2017domain}), bespoke data for training (e.g.,
\citealp{qiu2020fusing, karra2021global}), or both (e.g.,
\citealp{taubenbock2020, zhu2022urban, sharma2017patch, wang2018multi}).
% Transfer learning
Other works however rely on existing architectures like VGG/16
\citep{simonyan2014very}, UNet \citep{ronneberger2015u}, or ResNet
\citep{he2016deep}; standardised databases such as ImageNet \citep{ILSVRC15} for
training; or both (e.g., \citealp{qiu2020fusing,
karra2021global, srivastava2019understanding}). The former is known as \textit{transfer learning} and
usually involves re-training of the top layers of the network to customise
predictions to the specific use case, while retaining unchanged the original
weights for all the other layers.
%% List from MF on examples
%% https://github.com/urbangrammarai/signature_ai_paper/issues/5

            %%%%%%%%%%%%%%%

% RS focusing on urban env are of three kinds
    % LCLZ whose focus is not primarily on urban and only a fraction of their classes are urban
    % Built/unbuilt type of classifications (include JRC with 4 levels)
    % LCZ with pre-defined classes with an RS application in mind
% The understanding on how much of urban insight can we see from above is still very limited
% This paper takes a classification that 1) flips the ratio of urban/non-urban classes compared to
% LCLZ; 2) is data driven, with no particular secondary use case in mind (i.e. not designed to be seen);
% and asks the following questions:
    % can we reliably detect this type and granularity of urban classification from open satellite imagery?
    % given the geographical nature of the taks, is there a scope for spatially-explicit methods built on top of whatever is neural net doing?
    % would this potentially allow us to use AI and ML to potentially rollout time series of signatures based on Sentinel 2?

While deep learning has been recently introduced in the analysis of urban satellite
imagery, its application has so far mostly ignored the geographical nature of
the images being fed to these algorithms. This is not entirely unreasonable.
Much of the state of the art in deep learning and computer vision was
developed in the last decade with ``aspatial imagery'' in mind, in particular consumer
photographs uploaded and shared through the internet (e.g., featuring cats and
dogs). As such, many of the assumptions (e.g., unrelated images), tricks
(e.g., data augmentation techniques), and limitations (e.g., shape of the
input data) these models
feature are intimately related to data of this kind. The application of deep
learning to satellite imagery is in what we consider a first phase in which
cutting edge computer vision has been deployed to images that, rather than
animals or people, represent locations on Earth observed from above. Because
of the overall impressive performance of modern algorithms, the results are
impressive, even with largely unmodified models. However, this does not imply
there is no further margin for improvement.

No matter if the method is using the latest neural network or more traditional remote
sensing methods, the applications share a common limitation - they oversimplify urban
environment. In the most simple cases, we can talk about differentiation between
built-up areas and non-built-up land, occasionally expanded to a few gradual steps in
between (JRC). Land use and land cover classifications focus primarily on non-urban
landscape, leaving all cities in a handful of classes. Finally, LCZ classification is
pre-defined and designed with an RS application in mind. The understanding on how much
of "generic" urban insight can be seen from above is still very much an unexplored area.

This paper advances our understanding on how far can we go in application of
conventional deep learning methods to satellite imagery in a pursuit of capturing
composition of primarily urbanised landscape. It starts from the existing classification
of Great Britain that is data-driven, designed with no particular secondary use case in
mind (i.e. it is not designed to be seen on satellite imagery), and that flips the ratio
of urban vs non-urban classes compared to most LCLZ classifications. From there, we move
towards answering whether we can reliably detect this type and granularity of urban
classification from open satellite imagery. Furthermore, given the geographical nature
of the task, is there a scope for spatially-explicit methods built on top of the output
of neural network to bridge the gap between computer vision and geography? Understanding
these topics can potentially allow development of a time series of otherwise static
classifications like the one used in this paper and uncover the evolution of urban form
and function in cities.

%
The remainder of the paper is structured as follows:
Section \ref{sec:matmet} describes the data we use as well as the
methodological strategy we follow;
Section \ref{sec:results} presents the key results from our experiments;
and Section \ref{sec:discussion} discusses their relevance and concludes.

%% Much of the research above is focused on directly deploying standard computer imagery algorithms
%% The aspect of Geography is largely ignored
%% What do we mean by "the aspect of Geography"?
%%% How to treat images that represent geographical locations
%%% Some key differences with traditional computer vision:
%%% - images are "arbitrarily" cut from a continuous one --> role of scale
%%% - images are also thus intrinsically connected to each other through their spatial configuration --> role of context
%% Ignoring this means we are leaving "value on the table" when analysing satellite imagery
%% and standard computer vision does not provide off-the-shelf approaches

% This paper
%% Focus on the geographical aspect to explore its role in identifying UFF from satellites
%% We use state of the art AI/computer vision, but only as the starting point to explore geography
%% We set up a set of experiments that allow us to test a series of hypotheses on the role of scale and context
% Results
%% Scale and context matter, and these insights can be incorporated to improve predictions
% The remainder of this paper is structured as follows

%------------------------------------------------------------------------------------
%X Keep it conceptual about the point of the paper

%X Include literature review
%X (focused on what is available at the intersection of satellite + AI)

%X Highlight what the key missing bits are when AI is applied to spatial
%X imagery (e.g., scale and context)

%X a lot of the ai and satellite has been supervised object detection
%X there is not a lot of ai for patterns rather than features - in some way it may be easier
