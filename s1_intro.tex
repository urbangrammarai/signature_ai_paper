\section{Introduction}
\label{sec:intro} % ca. 1500 words

% Urban form and function
%% Important to understand
How different urban functions are arranged within space, and the forms these
give rise to is important to understand how cities work, how they
interact with the human and environmental systems that create them, and how
policy can effectively intervene.
%% Why? Encodes history, conditions the future
Urban form and function matter, at least, for two reasons \citep{dab_mf_2021a}:
first because cities use both to encode their history; and second because,
once in place, the physical layout of functions within a city condition how it
can and will develop in the future.
%% Key to understanding is measurement
A key requirement to understand form and function in cities is adequate
measurement, which implies detailed, consistent, and scalable
characterisations that can be updated frequently over time. These
characteristics then allow not only to observe detail, but to see it unfold
both over space and time. 
%% Rare and incomplete currently, of detailed, scalable and consistent, pick any two
There is a large literature measuring these phenomena, and it is relatively
common to find any two of those characteristics (i.e., detailed and consistent,
consistent and scalable, and detailed and scalable) present in a given piece
of work.
%% Recent developments are changing this --> spatial signatures
Research bringing the three together is still rare, although some is
emerging (e.g., \citealp{fleischmann2022geographical}) thanks to the confluence of better data, open
source software, and cheap computing power.
%% Detailed measurement is expensive in terms of time, effort, and data requirements
Still, generating detailed, consistent, and scalable classifications of urban
form and function is an expensive process that is difficult to refresh
regularly because most of the underlying sources only see updates
infrequently.

% The promise of satellites
%% A promising solution is to _supplement_ detailed measurements with satellite imagery
A promising option to improve the frequency of these classifications is
satellite imagery.
%% Satellite has radically improved in the last years, and it is set to continue on that technological path
Satellite technology has radically increased and improved the amount of
imagery available on the Earth, and shows no signs of slowing down.
%% At the same time, the algorithms to process satellite have also seen a revolution in the last ten years (deep learning)
More and better imagery has been complemented with the rise of new computer
vision algorithms, such as deep learning \citep{lecun2015},
that allow to extract more value from the same amount of data; and the
availability of computing power that makes it possible to deploy them cheaply
without the steep learning curve required only a few years ago.
%% These two trends converge in making possible things with satellite that was unthinkable a few years ago
The convergence of these two trends in remote sensing is unlocking
achievements that even very recently seemed beyond the realm of possibility.
%% Such as measuring UFF using and end-to-end open pipeline (data and code)
One such area is the use of remote sensing and satellite technology to decode
complex patterns in urban landscapes, such as the spatial signature of
different types of form and function.
%% Open is important because it multiplies the options of what is possible to do with outputs
Just as importantly, most of these developments are being contributed to the
community under open licenses that allow to further build on them and freely
redistribute downstream outputs.

            % Lit. review %
% Satellites for cities
The use of satellite technology for measuring different aspects of urban
environments is by no means new.
%% Mostly through Remote Urban Sensing
Much of the present work falls within the
broad category of urban remote sensing \citep{rashed2010remote, weng2018urban,
yang2021urban}. In fact, the promise of using remote sensing data to decode
the complexity of urban structure has long been recognised (e.g.,
\citealp{longley2002geographical}).
%% The vast majority is as supervised object detection --> include building footprints
%% And as Land Use and Land Cover --> Include recent examples (ESRI land cover, Google Dynamic World, etc.)

% Satellites for urban form and function
%% Much less on recognising composite patterns (e.g., UFF) instead of single objects or uses
%% In some ways, this is more complicated but, in others, maybe not (plus we have much better tools now!)
%% Much focused on Local Climate Zones

% Deep Learning Architectures
%% List from MF on examples
%% https://github.com/urbangrammarai/signature_ai_paper/issues/5

            %%%%%%%%%%%%%%%

% Research gap
%% Much of the research above is focused on directly deploying standard computer imagery algorithms
%% The aspect of Geography is largely ignored
%% What do we mean by "the aspect of Geography"?
%%% How to treat images that represent geographical locations
%%% Some key differences with traditional computer vision:
%%% - images are "arbitrarily" cut from a continuous one --> role of scale
%%% - images are also thus intrinsically connected to each other through their spatial configuration --> role of context
%% Ignoring this means we are leaving "value on the table" when analysing satellite imagery
%% and standard computer vision does not provide off-the-shelf approaches

% This paper
%% Focus on the geographical aspect to explore its role in identifying UFF from satellites
%% We use state of the art AI/computer vision, but only as the starting point to explore geography
%% We set up a set of experiments that allow us to test a series of hypotheses on the role of scale and context
% Results
%% Scale and context matter, and these insights can be incorporated to improve predictions
% The remainder of this paper is structured as follows

%------------------------------------------------------------------------------------
%X Keep it conceptual about the point of the paper

%X Include literature review
%X (focused on what is available at the intersection of satellite + AI)

%X Highlight what the key missing bits are when AI is applied to spatial
%X imagery (e.g., scale and context)

%X a lot of the ai and satellite has been supervised object detection
%X there is not a lot of ai for patterns rather than features - in some way it may be easier
