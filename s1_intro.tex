\section{Introduction}
\label{sec:intro} % ca. 1500 words

% Urban form and function
%% Important to understand
%% Why? Encodes history, conditions the future
%% Key to understanding is measurement
%% Rare and incomplete currently, of detailed, scalable and consistent, pick any two
%% Recent developments are changing this --> spatial signatures

% The promise of satellites
%% Detailed measurement is expensive in terms of time, effort, and data requirements
%% A promising solution is to _supplement_ detailed measurements with satellite imagery
%% Satellite has radically improved in the last years, and it is set to continue on that technological path
%% At the same time, the algorithms to process satellite have also seen a revolution in the last ten years (deep learning)
%% These two trends converge in making possible things with satellite that was unthinkable a few years ago
%% Such as measuring UFF using and end-to-end open pipeline (data and code)
%% Open is important because it multiplies the options of what is possible to do with outputs

            % Lit. review %
% Satellites for cities
%% Mostly through Remote Urban Sensing
%% The vast majority is as supervised object detection --> include building footprints
%% And as Land Use and Land Cover --> Include recent examples (ESRI land cover, Google Dynamic World, etc.)

% Satellites for urban form and function
%% Much less on recognising composite patterns (e.g., UFF) instead of single objects or uses
%% In some ways, this is more complicated but, in others, maybe not (plus we have much better tools now!)
%% Much focused on Local Climate Zones

% Deep Learning Architectures
%% List from MF on examples
%% https://github.com/urbangrammarai/signature_ai_paper/issues/5

            %%%%%%%%%%%%%%%

% Research gap
%% Much of the research above is focused on directly deploying standard computer imagery algorithms
%% The aspect of Geography is largely ignored
%% What do we mean by "the aspect of Geography"?
%%% How to treat images that represent geographical locations
%%% Some key differences with traditional computer vision:
%%% - images are "arbitrarily" cut from a continuous one --> role of scale
%%% - images are also thus intrinsically connected to each other through their spatial configuration --> role of context
%% Ignoring this means we are leaving "value on the table" when analysing satellite imagery
%% and standard computer vision does not provide off-the-shelf approaches

% This paper
%% Focus on the geographical aspect to explore its role in identifying UFF from satellites
%% We use state of the art AI/computer vision, but only as the starting point to explore geography
%% We set up a set of experiments that allow us to test a series of hypotheses on the role of scale and context
% Results
%% Scale and context matter, and these insights can be incorporated to improve predictions
% The remainder of this paper is structured as follows

%------------------------------------------------------------------------------------
%X Keep it conceptual about the point of the paper

%X Include literature review
%X (focused on what is available at the intersection of satellite + AI)

%X Highlight what the key missing bits are when AI is applied to spatial
%X imagery (e.g., scale and context)

%X a lot of the ai and satellite has been supervised object detection
%X there is not a lot of ai for patterns rather than features - in some way it may be easier